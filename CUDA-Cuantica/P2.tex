% Created 2021-10-06 mar 13:05
% Intended LaTeX compiler: pdflatex
%%% Local Variables:
%%% LaTeX-command: "pdflatex --shell-escape"
%%% End:
\documentclass[11pt]{article}
\usepackage[utf8]{inputenc}
\usepackage[T1]{fontenc}
\usepackage{graphicx}
\usepackage{grffile}
\usepackage{longtable}
\usepackage{wrapfig}
\usepackage{rotating}
\usepackage[normalem]{ulem}
\usepackage{amsmath}
\usepackage{textcomp}
\usepackage{amssymb}
\usepackage{listings}
\usepackage{capt-of}
\usepackage{hyperref}
\hypersetup{colorlinks=true, linkcolor=black}
\setlength{\parindent}{0in}
\usepackage[margin=1.1in]{geometry}
\usepackage[spanish]{babel}
\usepackage{mathtools}
\usepackage{palatino}
\usepackage{fancyhdr}
\usepackage{sectsty}
\usepackage{engord}
\usepackage{cite}
\usepackage{graphicx}
\usepackage{setspace}
\usepackage[compact]{titlesec}
\usepackage[center]{caption}
\usepackage{placeins}
\usepackage{tikz}
\usetikzlibrary{positioning}
\usetikzlibrary{bayesnet}
\usetikzlibrary{shapes.geometric}
\usetikzlibrary{decorations.text}
\usepackage{color}
\usepackage{amsmath}
\usepackage{minted}
\usepackage{pdfpages}

\def\inline{\lstinline[basicstyle=\ttfamily,keywordstyle={}]}

\titlespacing*{\subsection}{0pt}{5.5ex}{3.3ex}
\titlespacing*{\section}{0pt}{5.5ex}{1ex}
\author{José Antonio Álvarez Ocete\\ Francisco Javier Sáez Maldonado}
\date{\today}
\title{Introducción a Hadoop y Spark\\\medskip
\large Procesamiento de Datos a Gran Escala}

\newcommand{\I}{\mathbb{I}}
\newcommand{\ra}{\rangle}
\newcommand{\la}{\langle}

\begin{document}

\maketitle

\tableofcontents

\section{Parte 1: Programación GPGPU}

\subsection{Recursos de la GPU}
\subsection{Suma de 2 vectores}
\subsection{Suma de 2 matrices}
\subsection{Stencil1d: Estudiar el efecto de la memoria compartida}

\section{Parte 2: Programación con QisKit: (Computación cuántica)}

Para esta parte de la práctica utilizaremos el \href{https://quantum-computing.ibm.com/composer}{Quantum Composer de IBM}. Puesto que el tenemos un número limitado de procesos a ejecutar (únicamente 5) veremos los resultados en el simulador sin llegar a medirlo en muchos casos.

\subsection{Puertas Cuánticas}

TODO: Añadir enunciado


\paragraph*{Puerta CNOT}

La única operación no trivial aplicable sobre un único bit es la negación: la puerta NOT. De la misma forma, es natural preguntarse cuál es el equivalente a la puerta NOT en el mundo cuántico. Dado que un qubit está descrito por dos amplitudes $\alpha$ y $\beta$:

\[
	|\varphi\la = \alpha |0\la + \beta |1\la,
\]

la puerta NOT será un intercambio entre las posiciones de estas amplitudes, obteniéndose así:

\[
	|\varphi\la = \beta |0\la + \alpha |1\la.
\]

La matriz unitaria que describe esta transformación es sencilla:

\[
X = \frac{1}{\sqrt 2}
\begin{pmatrix}
	0 & 1 \\
	1 & 0 
\end{pmatrix}
\]

Vemos la implementación de esta puerta en el Quantum Composer de IBM:

\begin{figure}[H]
	\centering
	\includegraphics[scale=0.8]{figures/gate-x.png}
	\caption{Circuito con puerta de Hadamard}
\end{figure}

Recordemos que en el Quantum Composer de IBM todos los qubits empiezan siempre en el estado $|0\la$. Tras aplicarlo a nuestro qubit una puerta $X$ obtendremos $|1\ra$.

\paragraph*{Puertas de Pauli}



\paragraph*{Puerta Hadamard}

Finalmente presentamos la puerta de Hadamard para un único bit. Está descrita por la siguiente matriz unitaria:

\[
	H_1 = \frac{1}{\sqrt 2}
	\begin{pmatrix}
		1 & 1 \\
		1 & -1 
	\end{pmatrix}
\]

Uno de sus usos más comunes es la superposición de qubits. Si aplicamos esta puerta al estado $|0\la$ obtenemos el estado de Bell:

\[
	H|0\la = \frac{1}{\sqrt 2} |0\la + \frac{1}{\sqrt 2} |1\la = |+\la
\]

Mientras que si se la aplicamos al estado $|1\la$ obtenemos:

\[
	H|1\la = \frac{1}{\sqrt 2} |0\la - \frac{1}{\sqrt 2} |1\la = |-\la
\]

Que también supone una superposición exacta de $|0\la$ y $|1\la$ puesto que $|1/\sqrt 2|^2 = |-1/\sqrt 2|^2 = 1/2$.

Podemos estudiar el comportamiento de esta puerta utilizando el Quantum Composer de IBM:

\begin{figure}[H]
	\centering
	\includegraphics[scale=0.8]{figures/gate-hadamard.png}
	\caption{Circuito con puerta de Hadamard}
\end{figure}

Mirando tanto la \href{URL}{esfera de Bloch} como las probabilidades vemos que tenemos la misma probabilidad de medir $0$ y $1$.

\subsection{Generación de números aleatorios con un Computador Cuántico}

TODO: añadir enunciado

Sabemos que utilizando la puerta de Hadarmad $H$ explicada en el apartado anterior ponemos un qubit $|0\la$ en superposición:

\[
	H|0\la = \frac{1}{\sqrt 2} |0\la + \frac{1}{\sqrt 2} |1\la
\]

Si ahora medimos este qubit obtendremos $|0\la$ con probabilidad $|1/\sqrt 2|^2 = 1/2$, y $|1\ra$ con probabilidad $1/2$. Esto es, hemos creado un generador de bits aleatorios utilizando un único qubit. Para crear un generador de 3 bits utilizaremos un sistema de 3 qubits. Inicialmente en el estado $|000\ra$, aplicaremos una puerta Hadamard a cada qubit de forma independiente, poniendo así cada qubit en superposición:

\[
	\hat H_3|000\la = \frac{|000\la + |001\la + |010\la + |011\la + |100\la + |101\la + |110\la + |111\la}{\sqrt 8} |0\la
\]

Donde la puerta $H_3$ tranformación de Hadamard para tres qubits. Se puede definir recursivamente de la siguiente forma:

\[
	H_m = H_1 \times H_{m-1}, H_1 = \frac{1}{\sqrt 2}
	\begin{pmatrix}
		1 & 1 \\
		1 & -1 
	\end{pmatrix}
\]

Así obtenemos la puerta Hadamard para tres qubits:

\[
H_3 = \frac{1}{2^{3/2}}
\begin{pmatrix}
	1 & 1 & 1 & 1 & 1 & 1 & 1 & 1 \\
	1 & -1 & 1 & -1 & 1 & -1 & 1 & -1 \\
	1 & 1 & -1 & -1 & 1 & 1 & -1 & -1 \\
	1 & -1 & -1 & 1 & 1 & -1 & -1 & 1 \\
	1 & 1 & 1 & 1 & -1 & -1 & -1 & -1 \\
	1 & -1 & 1 & -1 & -1 & 1 & -1 & 1 \\
	1 & 1 & -1 & -1 & -1 & -1 & 1 & 1 \\
	1 & -1 & -1 & 1 & -1 & 1 & 1 & -1
\end{pmatrix}
\]

Pasamos a realizar un estudio empírico del circuito diseñado. Lo implementamos utilizando el Quantum Composer de IBM:

\begin{figure}[H]
	\centering
	\includegraphics[scale=0.8]{figures/barplot.pdf}
	\caption{Comparación de número de veces que aparece cada palabra con y sin preprocesamiento.}
\end{figure}

Este simple circuito cuántico pone todos los qubits en superposición y después mide el resultado. Comprobamos los resultados ejecutando un trabajo con este circuito en el ordenador cuántico de IBM:

\begin{figure}[H]
	\centering
	\includegraphics[scale=0.8]{figures/barplot.pdf}
	\caption{Comparación de número de veces que aparece cada palabra con y sin preprocesamiento.}
\end{figure}


\subsection{Entrelazamiento}

\subsection{Sumador de 2 qbits}

\section{Ejercicios opcionales de la práctica 2}

\end{document}
